
\include{command}
\rmfamily
\section{Small Amplitude Pendulum}
\subsection{Data Analysis}
\begin{figure}[H]
	\centering\includegraphics[width=10cm]{at}
	\caption{$\alpha$-$t$ graph at a small amplitude ($<0.35$ rad)}
	\label{$Frequency-Intensity$ Graph for $70$cm tube}
\end{figure}
The experimental data is well fitted ($R^2=0.9966$) with the sinusoidal (Amp $= 0.3401$ rad·s$^{-2}$;  $T=4.03 \text{ sec}$). We obtain the following from experiment:
$$
2T_0=8.08\pm 0.02 \text{ sec};\ 
\theta_0=0.141\pm 0.001 \text{ rad}
.
$$
The theoretical values are
$$
\begin{aligned}
T_0&=4.04\pm 0.01\text{ sec};\\
\text{Amp}_{\text{theo}}&=
A^2\theta_0\\&=\frac{4\pi^2\theta_0}{T_0^2}\\&=0.3410\pm0.0029\text{ rad·s}^{-2}.
\end{aligned}
$$
A nice agreement can be seen from the following table
$$
\begin{tabular}{|c|c|c|}
	\hline 
	& 预测  & Experiment (Curve Fitting) \\ 
	\hline 
Amplitude (rad·s$^{-2}$)	& 4.04 $\pm$ 0.01  & 4.03 \\ 
	\hline
Period (sec)	& 0.3410 $\pm$ 0.0029 & 0.3401 \\ 
	\hline 
\end{tabular} 
$$
The minimal discrepancy confirms the validity of the theory.

\subsection{Question}
\begin{enumerate}
	\item \emph{Why is the minus sign in the expression of angular acceleration?}
	
	\textbf{Answer: }The angular acceleration is in the same direction as the restoring torque, which, is always in the opposite direction of the angular displacement.
\end{enumerate}


\section{Large Amplitude Pendulum}
\subsection{Data Analysis}
\begin{figure}[H]
	\centering\includegraphics[width=10cm]{lg}
	\caption{$\theta$-$t$ graph at a large amplitude ($\approx3$ rad)}
\end{figure}
At a large amplitude, the shape of the experimental data becomes more deviated from the fitted sinusoidal ($R^2=0.9925$). This suggest that the small-angle approximation fails to depict an accurate picture as $\theta_0$ approaches $\pi$ rad.

\subsection{Questions}
\begin{enumerate}
	\item \emph{Is this curve sinusoidal?}
	
	\textbf{Answer: }No, as noticable deviation is present from the fitted sinusoidal in Figure 2.
$$$$
	\item \emph{What is the displacement/acceleration corresponding to the turning point, when the speed is zero or maximum?}
	
	\textbf{Answer: }The displacement and acceleration are both zero when the velocity reaches maximum (in magnitude). When the velocity is zero, the displacement reaches maximum (in magnitude) while the acceleration is at a local minimum (in magnitude) (See Figure 3).
	\begin{figure}[H]
		\centering\includegraphics[width=13cm]{lgg}
		\caption{$\theta$-$\omega$-$\alpha$-$t$ graph at a large amplitude ($\approx 3$ rad)}
		\label{$Frequency-Intensity$ Graph for $70$cm tube}
	\end{figure}
\end{enumerate}



\end{document}
