
\include{command}
\rmfamily
\section{Data Analysis}
\subsection{$F$-$m$ Relationship}
\begin{figure}[H]
	\centering\includegraphics[width=7cm]{fm}
	\caption{Force-mass graph at $10$-cm radius in $2.4$-m·s$^{-1}$ tangential speed}
	\label{$Frequency-Intensity$ Graph for $70$cm tube}
\end{figure}
With both tangential speed and radius of the trajectory held constant, the centripetal force acting on an object is proportional to  the object's mass - that is, $$F\propto m\text{         with } v,r \text{ fixed.}$$


\subsection{$F$-${v}^2$ Relationship}
\begin{figure}[H]
	\centering\includegraphics[width=7cm]{fv}
	\caption{Force-speed$^2$ graph at $10$-cm radius with a $33.5$-g mass}
	\label{$Frequency-Intensity$ Graph for $70$cm tube}
\end{figure}
With both mass and radius of the trajectory held constant, the centripetal force acting on an object is proportional to  the object's tangential speed, but squared - that is, $$F\propto v^2\text{         with } m,r \text{ fixed.}$$


\subsection{$F$-$r$ Relationship}
\begin{figure}[H]
	\centering\includegraphics[width=7cm]{fr}
	\caption{Force-radius graph at $27$-rad·s$^{-1}$ angular speed with a $33.5$-g mass}
	\label{$Frequency-Intensity$ Graph for $70$cm tube}
\end{figure}
With both mass and angular speed held constant, the centripetal force acting on an object is proportional to  the radius of the trajectory - that is, $$F\propto r\text{         with } m,\omega \text{ fixed.}$$


\section{Discussion}
The results obtained in the analysis strongly support the theory $$
F = mv^2/r. $$
However, it is noticed that, depsite a similar slope of the graph obtained from the experiment to that of the prediction, the magnitude of the centripetal force is always smaller than the prediction within all procedures, resulting in a downward shift of the graph. The uniformity of shift present in Figure 2 and 3 suggests the interference	 of \emph{friction}. Specifically,
$$mv^2/r=F_{\text{pred}}=F_{\text{exp}}+F_{\text{fric}}$$
$$
F_{\text{exp}}=F_{\text{pred}}-F_{\text{fric}},
$$
explaining the uniform downward shift in Figure 2 and 3. The growing deviation in Figure 1 is also explained, as friction grows linearly in the normal force, which, in this experiment, grows linearly in the mass of the object.






\end{document}
